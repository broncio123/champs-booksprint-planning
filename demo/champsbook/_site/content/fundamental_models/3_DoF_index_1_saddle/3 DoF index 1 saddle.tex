\documentclass{article}
\usepackage{amsfonts, amsbsy, amssymb, amsmath, graphicx, float, subfigure}

\tolerance=5000 
\textwidth=16.6 cm 
\oddsidemargin=-.04 cm
\evensidemargin=-.04 cm 
\topmargin=-1.3 cm 
\textheight=22.9 cm




\newtheorem{definition}{Definition}
\newtheorem{assumption}{Assumption}
\newtheorem{hyp}{Hypothesis}
\newtheorem{theorem}{Theorem}
\newtheorem{lemma}{Lemma}[section]
\newtheorem{corollary}{Corollary}[section]
\newtheorem{proposition}{Proposition}[section]



\begin{document}







\section*{A Three Degree-of-Freedom Index One Saddle--A Saddle-Center-Center Equilibrium Point}



We consider  a quadratic 3 DoF  Hamiltonian:




\begin{equation}
H = \underbrace{\frac{\lambda}{2} \left(p_1^2 - q_1^2 \right)}_{H_1} + \underbrace{\frac{\omega_2}{2} \left(p_2^2 - q_2^2 \right)}_{H_2} + \underbrace{\frac{\omega_3}{2} \left(p_3^2 - q_3^2 \right)}_{H_3}, \quad \lambda, \, \omega_2, \, \omega_3 >0
\label{ham3}
\end{equation}

\noindent
with the corresponding Hamilton's equations given by:


\begin{eqnarray}
\dot{q}_1 & = & \frac{\partial H}{\partial p_1}= \lambda p_1, \nonumber \\
\dot{p}_1 & = & -\frac{\partial H}{\partial q_1}= \lambda q_1, \nonumber \\
\dot{q}_2 & = & \frac{\partial H}{\partial p_2}= \omega_2 p_2, \nonumber \\
\dot{p}_2 & = & -\frac{\partial H}{\partial q_2}= -\omega_2 q_2, \nonumber \\
\dot{q}_3 & = & \frac{\partial H}{\partial p_3}= \omega_3 p_3, \nonumber \\
\dot{p}_3 & = & -\frac{\partial H}{\partial q_3}= -\omega_3 q_3, 
\label{hameq3}
\end{eqnarray}

\noindent
These equations have an equilibrium point of saddle-center-center equilibrium type (index one saddle) at the origin.
Since the Hamiltonians $H_1$, $H_2$ and $H_3$ are uncoupled we can analyze the phase portraits for each separately. 
As in the previous examples, $H_1$ corresponds to the `'reactive mode''  (trajectories can become unbounded) and $H_2$ and $H_3$   are `'bath modes'' (trajectories are bounded). 

In this system reaction occurs when  the $q_1$ coordinate of a trajectory changes sign. Hence, as in the 2 DoF example, a `'natural'' dividing surface would be $q_1 =0$. This is a five dimensional surface in the six dimensional phase space. We want to examine its' structure more closely and, in particular, its intersection with a fixed 5 dimensional energy surface.





First, note that for reaction to occur we must have $H_1 >0$. Also, it is clear  from  the form of $H_2$ that $H_2 \ge 0$. Therefore, for reaction we must have $H = H_1 + H_2 >0$. The energy surface is given by:

\begin{equation}
\frac{\lambda}{2} \left(p_1^2 - q_1^2 \right) + \frac{\omega_2}{2} \left(p_2^2 + q_2^2, \right)  + \frac{\omega_3}{2} \left(p_3^2 + q_3^2, \right)= H_1 + H_2+ H_3 = H > 0, \quad H_1 > 0, \, H_2 \ge 0.
\label{2DoFES}
\end{equation}

\noindent
The intersection of $q_1=0$ with this energy surface is given by:



\begin{equation}
\frac{\lambda}{2} \, p_1^2  + \frac{\omega_2}{2} \left(p_2^2 + q_2^2, \right) + \frac{\omega_3}{2} \left(p_3^2 + q_3^2, \right)= H_1 + H_2 + H_3= H > 0, \quad H_1 > 0, \, H_2, \, H_3 \ge 0.
\label{2DoDS}
\end{equation}

\noindent
This is the isoenergetic DS. It has the form of a 3-sphere in the four dimensional $(q_1, p_1, q_2, p_2, q_3, p_3)$ space. It has two `'halves'' corresponding to the forward and backward reactions, respectively:

\begin{equation}
\frac{\lambda}{2} \, p_1^2  + \frac{\omega_2}{2} \left(p_2^2 + q_2^2, \right)  + \frac{\omega_3}{2} \left(p_3^2 + q_3^2, \right)= H_1 + H_2  + H_3= H > 0, \quad p_1 >0, \quad \mbox{forward DS},
\end{equation}

\begin{equation}
\frac{\lambda}{2} \, p_1^2  + \frac{\omega_2}{2} \left(p_2^2 + q_2^2, \right)  + \frac{\omega_3}{2} \left(p_3^2 + q_3^2, \right)= H_1 + H_2  + H_3= H > 0, \quad p_1 <0, \quad \mbox{backward DS}.
\end{equation}

\noindent
The forward and backward DS `'meet'' at  $p_1 =0$:

\begin{equation}
 \frac{\omega_2}{2} \left(p_2^2 + q_2^2, \right) + \frac{\omega_3}{2} \left(p_3^2 + q_3^2, \right)= H_2 + H_3 \ge  0,   \mbox{NHIM},
 \label{NHIM3D}
\end{equation}

\noindent
which is a normally hyperbolic invariant 3 sphere. It is {\em invariant} because on this set $q_1 = p_1 =0$ and, from \eqref{hameq3}, if $q_1 = p_1 =0$ the $\dot{q}_1 = \dot{p}_1 =0$. Hence, $q_1$ and $p_1$ always remain zero, and therefore   trajectories with these initial conditions always remain on \eqref{NHIM3D}. In other words, it is invariant. It is normally hyperbolic for the same reasons as for our 2 DoF example. The directions normal to \eqref{NHIM3D}, i.e. $q_1-p_1$, are linearized saddle like dynamics.





\end{document}


